\documentclass{ctexart}  % 使用支持中文的文档类
\usepackage{amsmath}
\usepackage{geometry}
\geometry{a4paper, margin=1in}

\begin{document}
\section{多元多维函数的隐函数定理应用}
考虑以下方程组:
\begin{align}  % 统一使用 {align} 而不是 { align }
F_1(x, y, z, u, v) &= x^2 + y^2 + z^2 - u^2 - v^2 = 0\\
F_2(x, y, z, u, v) &= xy + yz + zu - uv = 0
\end{align}
我们想在点 \(P_0 = (1, 1, 1, 1, 1)\) 附近,将 \(u\) 和 \(v\) 表示为 \(x\),\(y\) 和 \(z\) 的函数。
\subsection {隐函数定理的条件验证}
隐函数定理告诉我们,如果:
\begin {enumerate}
\item 函数 \(F_1\) 和 \(F_2\) 在点 \(P_0\) 附近具有连续偏导数。
\item 在点 \(P_0\) 处,对自变量 \(u\) 和 \(v\) 的雅可比行列式不为零。
\end {enumerate}
那么方程组可以解出 \(u = u(x, y, z)\) 和 \(v = v(x, y, z)\),并且这些函数在 \(P_0\) 附近有连续偏导数。
首先计算 \(F_1\) 和 \(F_2\) 对变量的偏导数:
\begin{align}
\frac{\partial F_1}{\partial x} &= 2x \
\frac{\partial F_1}{\partial y} &= 2y \
\frac{\partial F_1}{\partial z} &= 2z \
\frac{\partial F_1}{\partial u} &= -2u \
\frac{\partial F_1}{\partial v} &= -2v \
\end{align}
\begin{align}
\frac{\partial F_2}{\partial x} &= y \
\frac{\partial F_2}{\partial y} &= x + z \
\frac{\partial F_2}{\partial z} &= y + u \
\frac{\partial F_2}{\partial u} &= z - v \
\frac{\partial F_2}{\partial v} &= -u
\end{align}
在点 \(P_0 = (1, 1, 1, 1, 1)\) 处,所有这些偏导数都是连续的。
现在我们需要计算雅可比行列式:
\begin {align}
J = \begin {vmatrix}
\frac {\partial F_1}{\partial u} & \frac {\partial F_1}{\partial v} \
\frac {\partial F_2}{\partial u} & \frac {\partial F_2}{\partial v}
\end {vmatrix}
= \begin {vmatrix}
-2u & -2v \
z - v & -u
\end {vmatrix}
\end {align}
代入点 \(P_0 = (1, 1, 1, 1, 1)\) 的值:
\begin {align}
J = \begin {vmatrix}
-2 & -2 \
0 & -1
\end {vmatrix}
= (-2) \cdot (-1) - (-2) \cdot (0) = 2 \neq 0
\end {align}
因此,隐函数定理的条件满足,我们可以将 \(u\) 和 \(v\) 表示为 \(x\),\(y\) 和 \(z\) 的函数。
\subsection {隐函数的导数计算}
现在我们来计算 \(u\) 和 \(v\) 对 \(x\),\(y\) 和 \(z\) 的偏导数。
根据隐函数定理,我们有:
\begin{align}
\begin{pmatrix}
\frac{\partial u}{\partial x} & \frac{\partial v}{\partial x} \
\frac{\partial u}{\partial y} & \frac{\partial v}{\partial y} \
\frac{\partial u}{\partial z} & \frac{\partial v}{\partial z}
\end{pmatrix}
= -
\begin{pmatrix}
\frac{\partial F_1}{\partial u} & \frac{\partial F_1}{\partial v} \
\frac{\partial F_2}{\partial u} & \frac{\partial F_2}{\partial v}
\end{pmatrix}^{-1}
\begin{pmatrix}
\frac{\partial F_1}{\partial x} & \frac{\partial F_1}{\partial y} & \frac{\partial F_1}{\partial z} \
\frac{\partial F_2}{\partial x} & \frac{\partial F_2}{\partial y} & \frac{\partial F_2}{\partial z}
\end{pmatrix}
\end{align}
首先,我们需要计算雅可比矩阵的逆:
\begin {align}
\begin {pmatrix}
\frac {\partial F_1}{\partial u} & \frac {\partial F_1}{\partial v} \
\frac {\partial F_2}{\partial u} & \frac {\partial F_2}{\partial v}
\end {pmatrix}^{-1}
= \begin {pmatrix}
-2 & -2 \
0 & -1
\end {pmatrix}^{-1}
\end {align}
计算 \(2 \times 2\) 矩阵的逆:
\begin {align}
\begin {pmatrix}
-2 & -2 \
0 & -1
\end {pmatrix}^{-1}
= \frac {1}{(-2) \cdot (-1) - (-2) \cdot (0)}
\begin {pmatrix}
-1 & 2 \
0 & -2
\end {pmatrix}
= \frac {1}{2}
\begin {pmatrix}
-1 & 2 \
0 & -2
\end {pmatrix}
= \begin {pmatrix}
-\frac {1}{2} & 1 \
0 & -1
\end {pmatrix}
\end {align}
然后计算:
\begin {align}
\begin {pmatrix}
\frac {\partial F_1}{\partial x} & \frac {\partial F_1}{\partial y} & \frac {\partial F_1}{\partial z} \
\frac {\partial F_2}{\partial x} & \frac {\partial F_2}{\partial y} & \frac {\partial F_2}{\partial z}
\end {pmatrix}
= \begin {pmatrix}
2x & 2y & 2z \
y & x + z & y + u
\end {pmatrix}
\end {align}
在点 \(P_0 = (1, 1, 1, 1, 1)\) 处:
\begin {align}
\begin {pmatrix}
\frac {\partial F_1}{\partial x} & \frac {\partial F_1}{\partial y} & \frac {\partial F_1}{\partial z} \
\frac {\partial F_2}{\partial x} & \frac {\partial F_2}{\partial y} & \frac {\partial F_2}{\partial z}
\end {pmatrix}
= \begin {pmatrix}
2 & 2 & 2 \
1 & 2 & 2
\end {pmatrix}
\end {align}
现在我们可以计算偏导数矩阵:
\begin {align}
\begin {pmatrix}
\frac {\partial u}{\partial x} & \frac {\partial v}{\partial x} \
\frac {\partial u}{\partial y} & \frac {\partial v}{\partial y} \
\frac {\partial u}{\partial z} & \frac {\partial v}{\partial z}
\end {pmatrix}
&= -
\begin {pmatrix}
-\frac {1}{2} & 1 \
0 & -1
\end {pmatrix}
\begin {pmatrix}
2 & 2 & 2 \
1 & 2 & 2
\end {pmatrix} \
&= -
\begin {pmatrix}
-\frac {1}{2} \cdot 2 + 1 \cdot 1 & -\frac {1}{2} \cdot 2 + 1 \cdot 2 & -\frac {1}{2} \cdot 2 + 1 \cdot 2 \
0 \cdot 2 + (-1) \cdot 1 & 0 \cdot 2 + (-1) \cdot 2 & 0 \cdot 2 + (-1) \cdot 2
\end {pmatrix} \
&= -
\begin {pmatrix}
-1 + 1 & -1 + 2 & -1 + 2 \
-1 & -2 & -2
\end {pmatrix} \
&= -
\begin {pmatrix}
0 & 1 & 1 \
-1 & -2 & -2
\end {pmatrix} \
&=
\begin {pmatrix}
0 & -1 & -1 \
1 & 2 & 2
\end {pmatrix}
\end {align}
因此,在点 \(P_0 = (1, 1, 1, 1, 1)\) 处:
\begin {align}
\frac {\partial u}{\partial x} &= 0 \
\frac {\partial u}{\partial y} &= 1 \
\frac {\partial u}{\partial z} &= 1 \
\frac {\partial v}{\partial x} &= -1 \
\frac {\partial v}{\partial y} &= 2 \
\frac {\partial v}{\partial z} &= 2
\end {align}
\subsection {几何解释}
这个例子可以从几何角度理解为两个超曲面在五维空间中的交集:
\begin {align}
F_1 (x, y, z, u, v) &= x^2 + y^2 + z^2 - u^2 - v^2 = 0\
F_2 (x, y, z, u, v) &= xy + yz + zu - uv = 0
\end {align}
隐函数定理告诉我们,这个交集在 \(P_0 = (1, 1, 1, 1, 1)\) 点附近可以表示为一个三维曲面,这个曲面可以参数化为 \((x, y, z, u(x, y, z), v(x, y, z))\)。
我们刚才计算的偏导数告诉我们这个曲面在 \(P_0\) 点处的切平面方向。例如,当我们沿着 \(x\) 方向移动时:
\begin {align}
\Delta u &= \frac {\partial u}{\partial x} \Delta x = 0 \cdot \Delta x = 0\
\Delta v &= \frac {\partial v}{\partial x} \Delta x = -1 \cdot \Delta x = -\Delta x
\end {align}
这意味着,在 \(P_0\) 点附近,沿着 \(x\) 轴方向移动时,\(u\) 值基本保持不变,而 \(v\) 值以相同的量但方向相反地变化。
\end{document}
